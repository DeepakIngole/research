\prefacesection{Abstract}

Autonomous vehicle technologies offer potential to eliminate the number of traffic accidents
that occur every year, not only saving numerous lives but mitigating the costly economic and social
impact of automobile related accidents. The premise behind this dissertation is that autonomous cars of the near
future can only achieve this ambitious goal by obtaining the capability to successfully maneuver in friction-limited situations. 
 With automobile racing as an inspiration, this dissertation presents and experimentally validates three vital components
 for driving at the limits of tire friction. The first contribution is a feedback-feedforward steering algorithm  
 that enables an autonomous vehicle to accurately follow a specified trajectory at the friction limits while preserving
 robust stability margins. The second contribution is a trajectory generation algorithm that leverages the computational speed of 
 convex optimization to rapidly generate both a longitudinal speed profile and lateral curvature profile for the autonomous vehicle to follow.
 While the algorithm is applicable to a wide variety of driving objectives, the work presented is for the specific case of vehicle racing,
 and generating minimum-time profiles is therefore the chosen application. The final contribution is a set of iterative learning control and
 search algorithms that enable autonomous vehicles to drive more effectively by learning from previous driving maneuvers. These contributions enable an autonomous Audi TTS test vehicle to 
 drive around a race circuit at a level of performance comparable to a professional human driver. The dissertation concludes
 with a discussion of how the algorithms presented can be translated into automotive safety systems in the near future. 
 
 

%Such situations occur frequently in safety-critical situations caused by inclement weather, unpredicted obstacles, and driver
%inattentiveness.

