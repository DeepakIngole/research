\chapter{Conclusion}
\label{chapter6}

Inspired by automobile racing, this dissertation documented several contributions for trajectory planning and control at the limits of handling.
Chapter 2 presented a feedback-feedforward steering controller that simultaneously maintains vehicle stability at the limits
of handling while minimizing lateral path deviation. Section \ref{sec:baselineFFW} presented an initial baseline 
steering controller based on lookahead steering feedback and feedforward based on vehicle kinematics and steady-state tire forces. In \S \ref{sec:betafb}, analytical results revealed that path tracking performance
 of the baseline controller could be improved if the vehicle sideslip angle is held tangent to the desired path. This desired sideslip behavior was incorporated into the
feedforward control loop to create a robust steering controller capable of accurate path 
tracking and oversteer correction at the physical limits of tire friction (\S \ref{sec:goodFFW}). Experimental data collected from an 
Audi TTS test vehicle driving at the handling limits on a full length race circuit (\S \ref{sec:expresults}) demonstrated the desirable steering performance of the final controller design. 

Chapter 3 presented a fast algorithm for 
minimum-time path planning that divided the path generation task into two sequential lateral and longitudinal subproblems that were
solved repeatedly. The longitudinal subproblem, described in \S \ref{sec:VP}, determined the minimum-time velocity profile given a fixed curvature profile. The lateral subproblem updated the path given the fixed speed profile by solving a convex optimization problem 
 that minimized the curvature of the vehicle's driven path while staying within track boundaries and obeying discretized equations of motion (\S \ref{sec:UPDATE}).
 Experimental lap times and racing lines from the proposed method were shown to be comparable to both a nonlinear gradient
 descent solution and a trajectory recorded from a professional racecar driver (\S \ref{sec:ch3ExpRes}). The cost function for the path update subproblem was
 also improved by incorporating a distance minimization term in \S \ref{sec:ADDMINDIST}. 

Finally, Chapters 4 and 5 presented two approaches to gradually refine the driving performance of the autonomous race car 
over time. Chapter 4 developed two iterative learning control (ILC) formulations that gradually determined the proper steering 
and throttle inputs to precisely track the desired racing trajectory. In \S \ref{sec:ch4dsm}, simulation and analytical results were used to design and test convergence of 
 proportional-derivative (PD) and quadratically optimal (Q-ILC) iterative learning controllers. Experimental results at combined vehicle accelerations of 9 $\mathrm{m/s^2}$
 indicate that the proposed algorithm can rapidly attenuate trajectory following errors over just two or three laps of racing (\S \ref{sec:ch4ExpRes}).
 Chapter 5 presented a tree-search algorithm to minimize experimental lap times by learning different acceleration limits for each turn on the
track. An A* search algorithm was devised in \S \ref{sec:framingTL} to search through experimental data and find the best value of $\mu$ for each portion
 of the track in order to globally minimize the resulting lap time. Key developments of this algorithm include designing an appropriate A* heuristic 
 (\S \ref{sec:ch5astarheuristic}) to minimize the needed computation time and designing the cost function to account for the physical difficulty of altering the 
 vehicle's trajectory while understeering or oversteering. 


\section{Future Work}

The dissertation concludes with a discussion of both future work and applications of the research to commercial automotive safety
systems.

\subsubsection{Feedback-Feedforward Steering Controller}

One drawback of the feedback-feedforward steering controller in Chapter 2 is the reliance on steady-state feedforward estimates of 
vehicle states. This will cause issues for tracking highly transient trajectories, such as those encountered in obstacle
avoidance maneuvers \cite{joethesis}. Furthermore, because the sideslip dynamics are captured only at the \textit{feedforward} level
to ensure robust stability margins, the steering controller becomes sensitive to modeling errors
between the actual vehicle system and the steady-state model. 

There are several avenues for future work to improve the path tracking performance of the steering controller. One possibility to improve robustness
to plant modeling errors is to come back to the feedback controller that directly incorporates vehicle sideslip measurements in the feedback
control law (\ref{eqn:vveq}). This controller was shown to have excellent path tracking characteristics. However, the controller suffered from
poor stability margins at the handling limits, and a compromise was ultimately selected where steady-state predictions of the vehicle
sideslip were used instead. A promising solution is to use a blend of measured and predicted sideslip values. For example, a higher level controller could transition between measured or estimated vehicle sideslip in (\ref{eqn:vveq}) depending
on whether there is significant risk of the vehicle approaching the handling limits. 

A second avenue for future work is to eliminate transient path tracking errors by tightening the lanekeeping controller gains. Funke \cite{joethesis}
noted that transient dynamics become significant when avoiding obstacles at the limits of handling. An LQR approach for gain selection revealed that
tighter path tracking is possible if the gain on heading error $\Delta\Psi$ is significantly shortened. Understandably, the drawback of this tighter path tracking is higher
levels of steering input, typically resulting in high frequency twitches of the steering wheel. Again, an MPC controller could manage this
tradeoff between smoother steering inputs and path tracking error. More of the standard lookahead controller could be used in normal steady-state driving situations, and tighter
path tracking gains would be used in transient obstacle avoidance scenarios. 

\subsubsection{Rapid Path Generation}

The primary difficulty with the trajectory generation method from Chapter 3 is the need to balance minimizing path curvature and path length. 
Without the computational expense of directly minimizing lap time or applying a trial-and error method, it is difficult to determine
the areas of the track where minimizing distance is important. The presented method of learning optimization weights from human data provides a 
quick solution, but professional driver data is not
always available for a given racing circuit. To manage the tradeoff, it may be beneficial to develop an \textit{anytime} algorithm that
 starts by globally minimizing curvature with a cheap
convex optimization step, and then uses the remaining computational time to refine specific turns where minimizing curvature is unlikely to be the best approach. This would most likely rely
on general heuristics learned for racing in general, and not just for a specific track. For example, on sequences of alternating left/right turns, only minimizing curvature may not be the best approach. 

The second area for future work is transferring the presented algorithm onto an embedded computer for real-time trajectory planning. This enables
the controller to account for real-time changes such as competing race vehicles and updated estimates of tire friction. This requires
two steps. First, the convex optimization code for the path update step must be written in a language such as CVXGEN \cite{boydcvxgen}
that is suitable for real-time computing. Second, given hardware restrictions on the size of optimization problems for embedded computing,
the optimization algorithm must be modified into a ``preview" controller that optimizes the next several turns instead of the entire track. 
The feasibility of this approach has been confirmed with a preliminary analysis, which showed that an optimization over 500 meters of track could be completed
on the order of milliseconds using CVXGEN. 

\subsubsection{Iterative Driving Improvement}

Chapters 4 and 5 presented two interesting preliminary methods for an autonomous race car to learn how to drive better. Chapter 4 focused
on improving tracking of a desired trajectory via iterative learning control, while Chapter 5 focused on modifying the \textit{longitudinal}
component of the planned trajectory based on experimental observations of tire utilization and vehicle speeds. These two approaches should be
combined and tested together, so that the vehicle begins with a preliminary trajectory planned with a conservative assumed friction value, and
then slowly learns how to track that trajectory while simultaneously making the trajectory faster on the turns where tire slips are lower
than predicted. 

A key prerequisite for achieving this is the ability to perform the learning algorithms in real time. Both learning algorithms currently operate
offline after a lap (or several laps) have already been recorded, and typically take 30 - 60 seconds of computing time in MATLAB. C++ implementation
of the algorithms, particularly the tree-search approach from Chapter 5, could provide a significant computational speedup. Improvements in algorithm 
efficiency and parallelization would enable a system where learning is continuously performed on a separate processor during the autonomous run itself.
This would enable a futuristic system where the autonomous race vehicle could run uninterrupted for five or ten laps, improving the lap 
time each lap through iterative learning control and trajectory modification. 

\section{Applications for Future Automotive Safety \newline Systems}

Automobile racing is a fascinating subject, and the quest for an autonomous vehicle that can compete with the best human drivers is akin to 
to the search for chess algorithms in the 1970's and 80's that could defeat a grandmaster. While automobile racing occurs at
accelerations that are much higher than those seen on passenger highways, trajectory planning and control algorithms for an autonomous
race car have significant potential benefits for future autonomous passenger vehicles. In the same way that the race to beat the best
human chess players inspired a new generation of broadly applicable artificial intelligence techniques, designing a fast race vehicle
offers a new generation of technology for future passenger safety systems. In fact, transfer of technology from the race car
to the passenger automobile is nothing new, and everyday automotive technology ranging from direct-shift gearboxes to the modern 
disc brake can be traced to innovations in race technology \cite{deaton}. 

\subsubsection{Steering Controller}

The presented feedback-feedforward steering controller is immediately ready for application in commercial autonomous driving features.
The required inputs of the algorithm are relatively simple to obtain. The controller requires knowledge of a desired speed and curvature
profile, available from any high level trajectory planner that computes smooth driving profiles, such as the high level planner for Stanford's
2008 ``Junior" DARPA Urban Challenge Vehicle \cite{junior08}. Additionally, the controller requires knowledge of the vehicle error states,
namely the deviation from the desired path and the heading error from the desired path. While the Audi TTS used for experimental validation obtains these
precisely from DGPS technology, this technology is not viable for commercial driving. However, there has been a large research effort on vehicle
localization relative to a known map via sensor fusion of commercially available sensors such as standard GPS/INS, LIDAR, and cameras, resulting in
localization accuracy suitable for autonomous driving \cite{jo2015}. 

If incorporated in a passenger automobile, the feedback-feedforward algorithm would be able to achieve accurate and smooth driving
in non-emergency situations. Common issues frequently reported on autonomous vehicles such as steering wheel twitches could be avoided
along with significant lateral path deviation. Most importantly, in the event of an autonomous safety maneuver at the handling limits, the controller
could follow an emergency trajectory without losing stability or deviating off the desired trajectory into an obstacle. Furthermore, the steering
response would be smooth and non-oscillating, giving the human passengers confidence in the capability of their vehicle. 

\subsubsection{Rapid Trajectory Planner}

The rapid trajectory planner in Chapter 3 also offers potential for a commercial autonomous safety system. The algorithm could 
be reformulated as a high level trajectory planner for the next several hundred meters of open road instead of over an entire
closed-circuit race track. Combined with LIDAR or other sensor information on the presence of obstacles, the 
objective of the lateral planner could be to avoid an upcoming obstacle while staying on the road, a 
framework first proposed by Erlien et al. \cite{erlienACC} for shared human/computer control. However, instead of a shared controller minimizing deviation from the 
driver's steering command, this trajectory planner would autonomously avoid obstacles while attempting to
 maintain a minimum curvature path. The benefit of minimum-curvature obstacle avoidance is increased safety margins. By maximizing the permissible
 collision-free speeds the car can safely drive at, the envelope of safe driving trajectories is increased. Furthermore, in non-emergency scenarios,
 the trajectory planner can plan driving paths below the limits by driving through desired waypoints on the road with minimum curvature for
 driver comfort. 
 
 \subsubsection{Lap-to-Lap Learning}
  
 The algorithms for iteratively improving autonomous driving performance will be more difficult to apply in real-world situations, simply because most passenger driving doesn't
 occur over the same closed-circuit race course. However, there is an emerging trend towards automation in all aspects of society, and several
 industrial companies have expressed interest in iterative learning algorithms for repetitive driving maneuvers. For example, manufacturers of 
 agricultural equipment have sponsored iterative learning research for heavy vehicles that can precisely repeat the same driving pattern for applications such as fertilizer
and seed deployment \cite{liuILI}. Beyond agriculture, similar applications for iterative learning could include autonomous tour vehicles or 
industrial vehicles for applications such as mining. Additionally, many drivers travel on similar roads every day for commuting or other routine
trips. Iterative learning controllers could therefore be generalized for a network of cars to quickly detect important information about the road (e.g. path curvature, friction conditions) that could
be communicated back to upcoming vehicles. Advances in iterative learning could also enable applications where an automated system detects a routine commute and learns from human driver data
in order to replicate the driving style of a specific passenger over time. This would be a vital step in gaining acceptance of autonomous vehicles as passengers
would be more comfortable with a self-driving car that matches their own driving style. 